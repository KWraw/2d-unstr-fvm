\documentclass[paper=a4, fontsize=11pt]{ctexart}
% \setCJKmainfont[BoldFont={SimHei},ItalicFont={KaiTi}]{SimSun}
\usepackage{latexsym,bm}
\usepackage{xcolor}
\usepackage{graphicx}
\usepackage{amsmath}
\usepackage{indentfirst}
\usepackage{subfigure}
% \usepackage{subcaption}
\usepackage[utf8]{inputenc}
% \usepackage{notoccite}
% \bibliographystyle{unsrtnat}
% \usepackage[numbers,sort&compress]{natbib}

\setlength{\parindent}{2em}
\setlength{\baselineskip}{1.8em}
\setlength{\parskip}{1ex}

\usepackage[english]{babel} % English language/hyphenation
\usepackage{amsmath,amsfonts,amsthm} % Math packages
\usepackage{listings} % Required for inserting code snippets
\usepackage{fancyhdr} % Custom headers and footers
\usepackage{fancyvrb}
\usepackage{graphicx} % Include graphs
\usepackage{epsfig}
\pagestyle{fancyplain} % Makes all pages in the document conform to the custom headers and footers
\fancyhead{} % No page header - if you want one, create it in the same way as the footers below
\fancyfoot[L]{} % Empty left footer
\fancyfoot[C]{} % Empty center footer
\fancyfoot[R]{\thepage} % Page numbering for right footer
\renewcommand{\headrulewidth}{0pt} % Remove header underlines
\renewcommand{\footrulewidth}{0pt} % Remove footer underlines
\setlength{\headheight}{13.6pt} % Customize the height of the header

\numberwithin{equation}{section} % Number equations within sections (i.e. 1.1, 1.2, 2.1, 2.2 instead of 1, 2, 3, 4)
\numberwithin{figure}{section} % Number figures within sections (i.e. 1.1, 1.2, 2.1, 2.2 instead of 1, 2, 3, 4)
\numberwithin{table}{section} % Number tables within sections (i.e. 1.1, 1.2, 2.1, 2.2 instead of 1, 2, 3, 4)

%-------------------------------------------------------------------------------
%---------DEFINE THE PROPERTIES OF LISTINGS---------
%-------------------------------------------------------------------------------

\definecolor{DarkGreen}{rgb}{0.0,0.4,0.0} % Comment color
\definecolor{highlight}{RGB}{255,251,204} % Code highlight color

\lstdefinestyle{CPP}%
{%
  basicstyle          = \scriptsize\ttfamily,
  language            = C++,
  numbers             = left,
  numberstyle         = \tiny,
  % stepnumber          = 1,
  numbersep           = 5pt,
  backgroundcolor     = \color{gray},
  showspaces          = false,
  % showspaces          = true,
  showstringspaces    = false,
  showtabs            = false,
  % showtabs            = true,
  frame               = single,
  tabsize             = 2,
  captionpos          = b,
  breaklines          = true,
  breakatwhitespace   = false,
  morestring          = [b]",
  stringstyle         = \color{blue},
  keywordstyle        = \color{black},
  commentstyle        = \color{green!20},
  % identifierstyle     = \color{blue},
  moredelim           = **[is][\bfseries]{`}{`},
  fancyvrb            = true,
  breaklines=true,
}

\lstdefinestyle{Style1}{ % Define a style for your code snippet, multiple definitions can be made if, for example, you wish to insert multiple code snippets using different programming languages into one document
language=[95]Fortran, % Detects keywords, comments, strings, functions, etc for the language specified
backgroundcolor=\color{highlight}, % Set the background color for the snippet - useful for highlighting
basicstyle=\footnotesize\ttfamily, % The default font size and style of the code
breakatwhitespace=false, % If true, only allows line breaks at white space
breaklines=true, % Automatic line breaking (prevents code from protruding outside the box)
captionpos=b, % Sets the caption position: b for bottom; t for top
commentstyle=\usefont{T1}{pcr}{m}{sl}\color{DarkGreen}, % Style of comments within the code - dark green courier font
deletekeywords={}, % If you want to delete any keywords from the current language separate them by commas
%escapeinside={\%}, % This allows you to escape to LaTeX using the character in the bracket
firstnumber=1, % Line numbers begin at line 1
frame=single, % Frame around the code box, value can be: none, leftline, topline, bottomline, lines, single, shadowbox
frameround=tttt, % Rounds the corners of the frame for the top left, top right, bottom left and bottom right positions
keywordstyle=\color{Blue}\bf, % Functions are bold and blue
morekeywords={}, % Add any functions no included by default here separated by commas
numbers=left, % Location of line numbers, can take the values of: none, left, right
numbersep=10pt, % Distance of line numbers from the code box
numberstyle=\tiny\color{Gray}, % Style used for line numbers
rulecolor=\color{black}, % Frame border color
showstringspaces=false, % Don't put marks in string spaces
showtabs=false, % Display tabs in the code as lines
stepnumber=5, % The step distance between line numbers, i.e. how often will lines be numbered
stringstyle=\color{Purple}, % Strings are purple
tabsize=2, % Number of spaces per tab in the code
}

% Create a command to cleanly insert a snippet with the style above anywhere in the document
% insertcode_fortran
% insertcode_sh
% insertcode_emacs
\newcommand{\insertcode}[2]{\begin{itemize}\item[]\lstinputlisting[caption=#2,label=#1,style=Style1]{#1}\end{itemize}} % The first argument is the script location/filename and the second is a caption for the listing

\newcommand{\horrule}[1]{\rule{\linewidth}{#1}} % Create horizontal rule command with 1 argument of height

% For spectial character input
\newcommand{\BS}{\char`\\}

\usepackage[]{listings}

\usepackage{tikz}
\usetikzlibrary{calc}
\usepackage{ccaption}
% \renewcommand{\figurename}{Fig.}
\usepackage{caption}
% \captionsetup[wrapfigure]{name=Fig.}
\usepackage[]{hyperref}
\hypersetup{pdfborder=0 0 0}
\hypersetup{urlcolor=blue}
% color 在之前某个包中已经引用过了

% configuration of hyperlinks
\hypersetup{backref,
  % pdfpagemode=FullScreen,
  colorlinks=true,
  allcolors=blue
}